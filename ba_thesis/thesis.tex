\documentclass[11pt,a4paper]{report} 
% Alternativ für doppelseitigen Ausdruck (nur bei > 60 Seiten sinnvoll)
% \documentclass[11pt,a4paper,twoside,openright]{report} 

\include{preamble} % alle Pakete und Einstellungen

% Hier anpassen 
\newcommand{\welchethesis}{Bachelor}
% \newcommand{\welchethesis}{Master}
\newcommand{\thesisofwas}{of Science - B.Sc.}
\newcommand{\titel}{Study and implementation of a decentralized application that can provide
	permissionless financial services using an evm based blockchain}
%\newcommand{\kurztitel}{Template Abschlussarbeit}
\newcommand{\autor}{Mario Alberto Maita Orozco}
\newcommand{\datum}{30.06.2022} % Abgabedatum
\newcommand{\ort}{Wiesbaden}
\newcommand{\referent}{Prof.\ Dr.\ Eva-Maria Iwer}
\newcommand{\korreferent}{Prof.\ Dr.\ Marc-Alexander Zschiegner}

\begin{document}
\include{vorspann} % Titelseite, Erklärungen, etc.

\begin{abstract} 
\LaTeX\ bietet Buchdruckqualität für jedermann.
Wir zeigen anhand dieses durch persönliche Präferenzen geprägtes Template, 
wie man Buchdruckqualität für eine Abschlussarbeit einfach erreichen kann.
Dazu werden beispielhaft Lösungen zu üblichen Fragestellungen im Dokument 
vorgestellt.
Zunächst benötigt man ein passendes \LaTeX\ System mit einigen 
installierten Erweiterungspaketen, das es erlaubt das Template zu 
übersetzen. 
Neben den grundlegenden For\-ma\-tie\-rungs\-möglich\-keiten mit \LaTeX\ wird 
insbesondere das Erstellen und Einbinden von Grafiken, Listings und 
mathematischen Formeln gezeigt.
Des Weiteren werden Literatur- und andere Verzeichnisse eingebunden.
Nicht zuletzt finden sich auch sachdienliche Hinweise zum
Schreiben und Zitieren von Literatur.
\end{abstract}

\tableofcontents
\newpage 

\chapter{Introduction} \label{ch:intro}
%\epigraphhead[70]{\epigraph{Documentation is like sex: 
%when it is good, it is very, very good; and when it is bad, 
%sit is better than nothing.}{\textit{Dick Brandon}}}



\section{Motivation}
In the last years' blockchain technology has gained a lot of traction and many developers are willing to build applications on top of this technology.
The Ethereum network with the Ethereum Virtual Machine, or EVM\cite{evm}, (whose state is validated and copied by every node of the network) offers a new way of application development. Meaning that developers can now implement some functionality and deploy it to a public EVM blockchain, making the code/functionality tamper-resistant; in addition, the implemented functionality (or smart contract) can be called by any participant or other code/contracts within the network. This composability allows anyone to interact with already deployed contracts and build other applications on top of them. 
These kinds of applications are called Decentralized applications or Dapps\cite{dapps}, because they are built on top of a decentralized network such as  Ethereum.

\section{Goals}
The goals of this Thesis are:  study and provide a better understanding of how Dapps are built, which technology is used, and the development of a Dapp that uses well-known DeFi\cite{defi} protocols to let users access permissionless financial services such as lending and borrowing.

\section{Thesis Structure}
The Thesis is divided into the following chapters:
\begin{itemize}
	
	\item \textbf{Chapter ~\ref{ch:background} - Fundamentals}, gives a technical overwiev of the blockchain technology and the Ethereum network.
	\item \textbf{Chapter ~\ref{ch:soa}} ...
	\item \textbf{Chapter ~\ref{ch:appreq}} ...
	\item \textbf{Chapter ~\ref{ch:impl}} ...
	\item \textbf{Chapter ~\ref{ch:conclusion}} ...
\end{itemize}


%%%%%
%%%%% Chapter
\chapter{Fundamentals} \label{ch:background}

\section{Blockchain} \label{}
\subsection{Cryptography}
\subsubsection{Cryptographic hash functions}
\subsubsection{Merkle tree}
\subsection{Consensus Mechanism}
\subsection{Bitcoin}
asdfasdf~\cite{bitcoin}

\section{Ethereum}
\subsection{Ethereum Virtual Machine (EVM)}
\subsection{Smart Contracts}
\subsection{Decentralized Applications (Dapps)}

\begin{table}
\centering
\begin{tabular}{|l||l|l|}
\hline
\multicolumn{1}{|c|}{\textbf{Plattform}} & 
\multicolumn{1}{|c|}{\textbf{\LaTeX-Distribution}} & 
\multicolumn{1}{|c|}{\textbf{Editor}} \\\hline\hline
Linux/Unix & TeX Live & Texmaker \\\hline
MacOSX     & TeX Live & Texmaker \\\hline
Windows    & MiKTeX   & Texmaker \\\hline   
\end{tabular}
\caption{\LaTeX-Distributionen und Editor je Plattform}
\label{tab:disteditplattform}
\end{table}





\begin{figure}[htp]
\centering
\includegraphics[width=.9\textwidth]{images/cids}
\caption{Dateien zur Erstellung des Templates}
\label{fig:templateprozess}
\end{figure}



%%%%%
%%%%% Chapter
\chapter{State of the Art, Protocols to be used} \label{ch:soa}


%%%%%
%%%%% Chapter
\chapter{Application requirements} \label{ch:appreq}


\section{Listings} \label{sec:listings}

\begin{listing}[htbp]
\begin{lstlisting}
def ggt(x, y):
    while x != 0:
        x,y = y%x, x
    return y
\end{lstlisting}
\caption{ggT --- kurz und gut}
\label{code:ggt}
\end{listing}


%%%%%
%%%%% Chapter
\chapter{Implementation and testing} \label{ch:impl}
\newpage


%%%%%
%%%%% Chapter
\chapter{Conclusion and future work} \label{ch:conclusion}

\newpage

% Listen wenn überhaupt ans Ende und nicht an den Anfang.
% Meist ist das aber unnötig.
%\listoffigures % Liste der Abbildungen 
%\listoftables % Liste der Tabellen
% \newpage

\bibliographystyle{plain} % Literaturverzeichnis
\begin{btSect}{thesis} % mit bibtopic Quellen trennen
\section*{Bibliography}
\btPrintCited
\end{btSect}
\begin{btSect}{online}
\section*{Online Sources}
\btPrintCited
\end{btSect}
% dann mit "bibtex thesis1" und "bibtex thesis2" arbeiten

\end{document}
;;; Local Variables:
;;; ispell-local-dictionary: "de_DE-neu"
;;; End:
